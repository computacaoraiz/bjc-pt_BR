%%%%%%%%%%%%%%%%%%%%%%%%%%%%%%%%%%%%%%%%%%%%%%%%%%%%%%%%%%%%%%%%%%%%%%%%%%%%%%%%
\chapter{Prefácio}
\label{chap:prefacio}

Bem-vindo ao curso ``A Beleza e a Alegria da Computação'' (\ingles{Beauty and
Joy of Computing} --- BJC)!

Através deste currículo, você \textbf{criará} programas usando a linguagem de
programação \snap, aprenderá algumas das \textbf{idéias mais poderosas} da
ciência da computação e discutirá as implicações sociais da computação,
refletindo profundamente sobre como \textbf{você} pode ter um papel ativo na
promoção dos benefícios e na redução dos possíveis danos da computação.

Para ter a melhor experiência com o \snap, verifique se seu navegador está
atualizado.

Este curso está dividido nas seguintes unidades:

\begin{itemize}[noitemsep]
\item Unidade 1: Introdução à Programção
\item Undiade 2: Abstração
\item Unidade 3: Estruturas de Dados
      \begin{itemize}
      \item Tarefa de Criação
      \end{itemize}
\item Unidade 4: Como a Internet Funciona
\item Unidade 5: Algoritmos e Simulações
      \begin{itemize}
      \item Tarefa de Criação
      \end{itemize}
\item Unidade 6: Como os Computadores Funcionam
\item Unidade 7: Fractais e Recursão
\item Unidade 8: Funções Recursivas
\end{itemize}

As unidades 1--5 abrangem todo o currículo de 2020 do \ingles{Advanced Placement
Computer Science Principles} (AP CSP). Você está pronto para o exame\footnote{%
[NT] O \ingles{Advanced Placement Computer Science Principles} (AP CSP) é um
curso e um exame oferecido para estudantes no final do ensino médio americano e
canadense, como uma oportundiade de obter créditos extras e/ou dispensas em
disciplinas em cursos superiores. O curso/exame é oferecido pelo \ingles{College
Board}, uma organização não-lucrativa americana que promove currículos, padrões,
cursos e exames para estudantes em diversos níveis do ensino fundamental e médio
como forma de prepará-los para a entrada na universidade. O \ingles{College
Board} promove diversas iniciativas e exames, entre eles o já citado AP e o
famoso SAT (\ingles{Scholastic Aptitude Test}). Para saber mais:
\begin{itemize}[noitemsep]
\item \ingles{College Board}: \url{https://en.wikipedia.org/wiki/College_Board}
\item AP: \url{https://en.wikipedia.org/wiki/Advanced_Placement}
\item AP CSP: \url{https://en.wikipedia.org/wiki/AP_Computer_Science_Principles}
\item SAT: \url{https://en.wikipedia.org/wiki/SAT}
\end{itemize}.}. As unidades 6--8 focam na hierarquia de abstrações que permitem
que os computadores funcionem, bem como na recursão, uma idéia bonita e poderosa
da ciência da computação que vai além do currículo e do exame do AP CSP. Essas
unidades são perfeitas para aproveitar após o exame.

A Beleza e a Alegria da Computação (\ingles{Beauty and Joy of Computing}) é
reconhecida pelo \ingles{College Board} como uma fornecedora aprovada do
currículo e desenvolvimento profissional para o AP CSP. Esse reconhecimento
afirma que todos os componentes da Beleza e a Alegria da Computação estão
alinhados com os padrões do currículo e a avaliação AP CSP. O uso de um
fornecedor aprovado proporciona às escolas acesso a recursos, incluindo um plano
de curso AP CSP pré-aprovado pela auditoria de cursos AP do \ingles{College
Board}, e desenvolvimento profissional oficialmente reconhecido que prepara os
professores para ensinar o AP CSP.

A Beleza e a Alegria da Computação, da Universidade da California, em Berkeley,
e do \ingles{Education Development Center}, está licenciada sob 
`\ingles{Creative Commons Attribution-NonCommercial-ShareAlike 4.0 International
License 4.0}''\footnote{CC BY-NC-SA 4.0:
\url{https://creativecommons.org/licenses/by-nc-sa/4.0/}}.

\begin{figure}[h]
\centering
%\caption{}
%\label{fig:}
%\fbox{
  \includegraphics[scale=1.5]{imagens/bjc-logo.jpg}
%}\\
%\footnotesize{Fonte: xxx}
\end{figure}
